\documentclass{article}

% Packages
\usepackage{geometry}
\usepackage{graphicx}
\usepackage{lipsum}
\usepackage{listings}
\usepackage[portuguese]{babel}

% Page setup
\geometry{a4paper, margin=2cm}

\begin{document}

% cover
\thispagestyle{empty} % Remove page number from first page

\begin{figure}[t]
    \includegraphics[width=3cm]{images/logo-puc-minas.png}
    \hspace{0.02\textwidth}
    \vline%
    \hspace{0.04\textwidth}
    \includegraphics[width=3cm]{images/logo-icei.jpeg}
\end{figure}

\hrulefill%
\vspace{\baselineskip}

\Large\noindent
\textbf{Pontifícia Universidade Católica de Minas Gerais} \\
\textbf{Instituto de Ciências Exatas e Informática} \\
\textbf{Departamento de Engenharia de Computação}

\begin{center}
    \vfill
    \Huge\textbf{Relatório do Laboratório 5} \\
    \vspace{0.5cm}
    \Large\textbf{Fonte Regulada com Diodo Zener} \\
    \vspace{1cm}
    \large \textbf{Professor}: Bruno Luiz Dias Alves de Castro \\
    \vspace{0.5cm}
    \large Bruno Luiz Dias Alves de Castro \\
    \large Bruno Luiz Dias Alves de Castro \\
    \large Bruno Luiz Dias Alves de Castro \\
    \vfill
    \large Belo Horizonte \\ Campus Coração Eucarístico \\
    \vspace{\baselineskip}
    \large \today
\end{center}

% table of contents
\newpage
\thispagestyle{empty}
\tableofcontents

% body
\newpage
\large % document text size

\section{Introdução}

Durante as aulas da disciplina de Sistemas Reconfiguráveis, fomos introduzidos à linguagem VHDL. VHDL (\textbf{V}HSIC \textbf{H}ardware \textbf{D}escription \textbf{L}anguage) é uma linguagem de descrição de hardware. Com ela, podemos montar circuitos lógicos de maneira totalmente textual, o que garante à linguagem uma grande vantagem ante à soluções visuais.

\subsection{Objetivos}

\subsection{Simulação via Quartus II}

Nessa etapa realizamos testes no software Quartus II da altera.

\subsubsection{Bloco w\_reg}

Nesta imagem é realizado 3 testes para verificar a funcionalidade do registrador, nos primeiros 60ns é alterado os bits da entrada de dados (d\_in) para nivel lógico alto, o bit de reset (nrst) que é ativo em baixa, é desativado, ou seja, nível lógico alto e o bit de ativação (wr\_en) é colocoado em nível lógico alto após 10ns. Assim é possível verificar a mudança na saída (w\_out) com um tempo de delay de 6ns. No segundo teste a partir de 60ns até 140ns é resetado os bits da memória do registrador colocando reset em nível lógico zero, o resultado é propagada para a saída após o tempo de delay de aproximadamente 6ns. No terceiro teste foi verificados se o bit de ativação de escrita está funcionando corretamente, portanto com o bit 6 da saída em nível lógico alto esse valor será escrito apenas no tempo 160ns quando é colocado a porta de ativação do registrador em nível lógico alto e o registrador é escrito.

\begin{figure}[ht]
\begin{center}
    \includegraphics[width=15cm]{images/w_reg.png}
    \caption{Simulação bloco w\_reg}
\end{center}
\end{figure}

\section{Conclusão}
Com estes dois projetos simples, tivemos um excelente primeiro contado com a linguagem VHDL, bem como à programação concorrente e desenvolvimento de circuitos FPGA. Os dois circuitos implementados (Multiplexador de Endereçamento e Unidade Lógica Aritimética) são blocos de construção chave para a maior parte dos circuitos complexos, e serão de suma importância não só para os demais trabalhos práticos que realizaremos ao longo do semestre, mas para nosso desenvolvimento acadêmico e profissional.

\end{document}
