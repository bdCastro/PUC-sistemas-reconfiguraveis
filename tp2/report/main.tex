\documentclass{article}

% Packages
\usepackage{geometry}
\usepackage{graphicx}
\usepackage{lipsum}
\usepackage{listings}
\usepackage[portuguese]{babel}

% -- Defining colors:
\usepackage[dvipsnames]{xcolor}
\definecolor{codegreen}{rgb}{0,0.6,0}
\definecolor{codegray}{rgb}{0.5,0.5,0.5}
\definecolor{codepurple}{rgb}{0.58,0,0.82}
\definecolor{backcolour}{rgb}{0.95,0.95,0.92}

% Definig a custom style:
\lstdefinestyle{mystyle}{
    backgroundcolor=\color{backcolour},   
    commentstyle=\color{codepurple},
    keywordstyle=\color{NavyBlue},
    numberstyle=\tiny\color{codegray},
    stringstyle=\color{codepurple},
    basicstyle=\ttfamily\footnotesize\bfseries,
    breakatwhitespace=false,         
    breaklines=true,                                   
    keepspaces=true,                 
    numbers=left,                    
    numbersep=5pt,                  
    showspaces=false,                
    showstringspaces=false,
    showtabs=false,                  
    tabsize=2,
    captionpos=b
}

% -- Setting up the custom style:
\lstset{style=mystyle}

% Page setup
\geometry{a4paper, margin=2cm}

\begin{document}

% cover
\thispagestyle{empty} % Remove page number from first page

\begin{figure}[t]
    \includegraphics[width=3cm]{images/logo-puc-minas.png}
    \hspace{0.02\textwidth}
    \vline%
    \hspace{0.04\textwidth}
    \includegraphics[width=3cm]{images/logo-icei.jpeg}
\end{figure}

\hrulefill%
\vspace{\baselineskip}

\Large\noindent
\textbf{Pontifícia Universidade Católica de Minas Gerais} \\
\textbf{Instituto de Ciências Exatas e Informática} \\
\textbf{Departamento de Engenharia de Computação}

\begin{center}
    \vfill
    \Huge\textbf{Relatório do Laboratório 5} \\
    \vspace{0.5cm}
    \Large\textbf{Fonte Regulada com Diodo Zener} \\
    \vspace{1cm}
    \large \textbf{Professor}: Bruno Luiz Dias Alves de Castro \\
    \vspace{0.5cm}
    \large Bruno Luiz Dias Alves de Castro \\
    \large Bruno Luiz Dias Alves de Castro \\
    \large Bruno Luiz Dias Alves de Castro \\
    \vfill
    \large Belo Horizonte \\ Campus Coração Eucarístico \\
    \vspace{\baselineskip}
    \large \today
\end{center}

% table of contents
\newpage
\thispagestyle{empty}
\tableofcontents

% body
\newpage
\large % document text size

\section{Introdução}

Durante as aulas da disciplina de Sistemas Reconfiguráveis, fomos introduzidos à linguagem VHDL. VHDL (\textbf{V}HSIC \textbf{H}ardware \textbf{D}escription \textbf{L}anguage) é uma linguagem de descrição de hardware. Com ela, podemos montar circuitos lógicos de maneira totalmente textual, o que garante à linguagem uma grande vantagem ante à soluções visuais.

\subsection{Objetivos}

\subsection{Simulação via Quartus II}

Nessa etapa realizamos testes no software Quartus II da altera.

\section{w\_reg}

O registrador w\_reg é um registrador simples. Possui um barramento de dados para escrita, habilitada por um sinal wr\_in.\\

As entradas e saídas do circuito são descritas na tabela a baixo:\\

\begin{table}[ht]
    \begin{center}
        \begin{tabular}{|c|c|c|c|}
            \hline
            Nome & Tamanho & Tipo & Descrição\\
            \hline
            nrst & 1 bit & \textit{Input} & Entrada de \textit{reset} assíncrono.\\
            \hline
            clk\_in & 1 bit & \textit{Input} & Entrada de \textit{clock}.\\
            \hline
            d\_in & 8 bits & \textit{Input} & Entrada de dados para escrita.\\
            \hline
            wr\_en & 1 bit & \textit{Input} & Entrada de habilitação de escrita.\\
            \hline
            w\_out & 8 bits & \textit{Output} & Saída de dados.\\
            \hline
        \end{tabular}
    \end{center}
    \caption{Entradas e Saídas de fsr\_reg}
\end{table}

\subsection{Implementação}

O registrador w\_reg foi implementado utilizando a linguagem VHDL.\\

O código na íntegra está abaixo:\\

\begin{lstlisting}[language=VHDL, caption={Código VHDL w\_reg}]
LIBRARY ieee;
USE ieee.std_logic_1164.all;
USE ieee.std_logic_unsigned.all;
USE ieee.numeric_std.all;

ENTITY w_reg IS
    PORT (
        -- Inputs
        nrst : IN STD_LOGIC;                        -- Reset
        clk_in: IN STD_LOGIC;                       -- Clock
        d_in: IN STD_LOGIC_VECTOR(7 DOWNTO 0);      -- Dados
        wr_en : IN STD_LOGIC;                       -- Enable

        -- Outputs
        w_out : OUT STD_LOGIC_VECTOR(7 DOWNTO 0)    -- Dados
    );
END ENTITY;

ARCHITECTURE w_reg OF w_reg IS
    SIGNAL mem_reg: STD_LOGIC_VECTOR(7 DOWNTO 0);
BEGIN

    PROCESS (nrst, clk_in)
    BEGIN
        IF nrst = '0' THEN
            mem_reg <= "00000000";
        ELSIF RISING_EDGE(clk_in) then
            IF wr_en = '1' THEN
                mem_reg <= d_in;
            END IF;
        END IF;
    END PROCESS;

    w_out <= mem_reg;

END w_reg;
\end{lstlisting}

\subsection{Simulação}

Nesta imagem é realizado 3 testes para verificar a funcionalidade do registrador, nos primeiros 60ns é alterado os bits da entrada de dados (d\_in) para nivel lógico alto, o bit de reset (nrst) que é ativo em baixa, é desativado, ou seja, nível lógico alto e o bit de ativação (wr\_en) é colocoado em nível lógico alto após 10ns. Assim é possível verificar a mudança na saída (w\_out) com um tempo de delay de 6ns. No segundo teste a partir de 60ns até 140ns é resetado os bits da memória do registrador colocando reset em nível lógico zero, o resultado é propagada para a saída após o tempo de delay de aproximadamente 6ns. No terceiro teste foi verificados se o bit de ativação de escrita está funcionando corretamente, portanto com o bit 6 da saída em nível lógico alto esse valor será escrito apenas no tempo 160ns quando é colocado a porta de ativação do registrador em nível lógico alto e o registrador é escrito.

\begin{figure}[ht]
\begin{center}
    \includegraphics[width=15cm]{images/w_reg.png}
    \caption{Simulação bloco w\_reg}
\end{center}
\end{figure}

\newpage

\section{fsr\_reg}

O registrador FSR é um registrador semelhante ao implementado anteriormente. A principal diferença entre os dois está na presença de um sistema de endereçamento, e de duas entradas binárias independentes para habilitação da escrita e da leitura. Os requisitos são descritos na tabela abaixo.

\begin{table}[ht]
    \begin{center}
        \begin{tabular}{|c|c|c|c|}
            \hline
            Nome & Tamanho & Tipo & Descrição\\
            \hline
            nrst & 1 bit & \textit{Input} & Entrada de \textit{reset} assíncrono.\\
            \hline
            clk\_in & 1 bit & \textit{Input} & Entrada de \textit{clock}.\\
            \hline
            abus\_in & 9 bit & \textit{Input} & Entrada de enderençamento.\\
            \hline
            dbus\_in & 8 bits & \textit{Input} & Entrada de dados para escrita.\\
            \hline
            wr\_en & 1 bit & \textit{Input} & Entrada de habilitação de escrita.\\
            \hline
            rd\_en & 1 bit & \textit{Input} & Entrada de habilitação de leitura.\\
            \hline
            dbus\_out & 8 bits & \textit{Output} & Saída de dados hailitada por rd\_en.\\
            \hline
        \end{tabular}
    \end{center}
    \caption{Entradas e Saídas de fsr\_reg}
\end{table}

\subsection{Implementação}

O registrador fsr\_reg foi implementado utilizando a linguagem VHDL.\\

O código na íntegra está abaixo:\\

\begin{lstlisting}[language=VHDL, caption={Código VHDL fsr\_reg}]
LIBRARY ieee;
USE ieee.std_logic_1164.all;
USE ieee.std_logic_unsigned.all;
USE ieee.numeric_std.all;

ENTITY fsr_reg IS
    PORT (
        -- Inputs
        nrst : IN STD_LOGIC;                            -- Reset
        clk_in: IN STD_LOGIC;                           -- Clock
        abus_in: IN STD_LOGIC_VECTOR(8 DOWNTO 0);       -- Enderecamento
        dbus_in: IN STD_LOGIC_VECTOR(7 DOWNTO 0);       -- Dados
        wr_en : IN STD_LOGIC;                           -- Enable escrita
        rd_en : IN STD_LOGIC;                           -- Enable leitura

        -- Outputs
        dbus_out : OUT STD_LOGIC_VECTOR(7 DOWNTO 0);    -- Dados
        fsr_out : OUT STD_LOGIC_VECTOR(7 DOWNTO 0)      -- Registrador
    );
END ENTITY;

ARCHITECTURE fsr_reg OF fsr_reg IS
    SIGNAL mem_reg: STD_LOGIC_VECTOR(7 DOWNTO 0);
BEGIN
    PROCESS (nrst, clk_in, mem_reg, abus_in, dbus_in)
    BEGIN
        IF nrst = '0' THEN
            mem_reg <= "00000000";
        ELSIF abus_in(6 DOWNTO 0) = "0000100" THEN
            IF RISING_EDGE(clk_in) THEN
                IF wr_en = '1' THEN
                    mem_reg <= dbus_in;
                END IF;
            END IF;
        END IF;
    END PROCESS;

    dbus_out <= mem_reg WHEN rd_en = '1' ELSE "ZZZZZZZZ";
    fsr_out <= mem_reg;
END fsr_reg;    
\end{lstlisting}

\subsection{Simulação}

Para testar nosso código VHDL e certificar-nos de que nosso circuito funciona de maneira esperada, simulamos alguns casos de testes utilizando o software Quatus II.\\

Os testes realizados foram os seguites:

\begin{enumerate}
    \item Escrita com enderaçamento incorreto (diferente de XX0000100).
    \begin{itemize}
        \item \textbf{Comportamento esperado:}
        \begin{itemize}
            \item dbus\_out em alta impedância;
            \item fsr\_out sem alteração;
        \end{itemize}
    \end{itemize}
    
    \item Leitura habilitada e escrita desabilitada.
    \begin{itemize}
        \item \textbf{Comportamento esperado:}
        \begin{itemize}
            \item dbus\_out = frs\_out = último valor escrito;
        \end{itemize}
    \end{itemize}

    \item Leitura desabilitada e escrita habilitada.
    \begin{itemize}
        \item \textbf{Comportamento esperado:}
        \begin{itemize}
            \item dbus\_out em alta impendância;
            \item frs\_out = dbus\_in;
        \end{itemize}
    \end{itemize}

    \item \textit{Reset} com leitura habilitada.
    \begin{itemize}
        \item \textbf{Comportamento esperado:}
        \begin{itemize}
            \item dbus\_out = frs\_out = ``0b00000000'';
        \end{itemize}
    \end{itemize}
\end{enumerate}

\begin{figure}[ht]
    \begin{center}
        \includegraphics[width=15cm]{images/fsr_reg.png}
        \caption{Simulação fsr\_reg}
\end{center}
\end{figure}

\newpage

\section{status\_reg}

O status é um registrador semelhante ao implementado anteriormente. A diferença esta na presença de sinais de entrada e saída para controlar bits específicos. Assim como o anterior, existe um sistema de endereçamento que deve ser conferido para alterar o registrador.\\

As entradas e saídas do circuito são descritas na tabela a baixo:\\

\begin{table}[ht]
    \begin{center}
        \begin{tabular}{|c|c|c|c|}
            \hline
            Nome & Tamanho & Tipo & Descrição\\
            \hline
            nrst & 1 bit & \textit{Input} & Entrada de \textit{reset} assíncrono.\\
            \hline
            clk\_in & 1 bit & \textit{Input} & Entrada de \textit{clock}.\\
            \hline
            abus\_in & 9 bit & \textit{Input} & Entrada de enderençamento.\\
            \hline
            dbus\_in & 8 bits & \textit{Input} & Entrada de dados para escrita.\\
            \hline
            wr\_en & 1 bit & \textit{Input} & Entrada de habilitação de escrita.\\
            \hline
            rd\_en & 1 bit & \textit{Input} & Entrada de habilitação de leitura.\\
            \hline
            z\_in & 1 bit & \textit{Input} & Entrada de dado para escrita no bit 2 do registrador.\\
            \hline
            dc\_in & 1 bit & \textit{Input} & Entrada de dado para escrita no bit 1 do registrador.\\
            \hline
            c\_in & 1 bit & \textit{Input} & Entrada de dado para escrita no bit 0 do registrador.\\
            \hline
            z\_wr\_en & 1 bit & \textit{Input} & Entrada para habilitação da escrita no bit 2 do registrador.\\
            \hline
            dc\_wr\_en & 1 bit & \textit{Input} & Entrada para habilitação da escrita no bit 1 do registrador.\\
            \hline
            c\_wr\_en & 1 bit & \textit{Input} & Entrada para habilitação da escrita no bit 0 do registrador.\\
            \hline
            dbus\_out & 8 bits & \textit{Output} & Saída de dados hailitada por rd\_en.\\
            \hline
            irp\_out & 1 bit & \textit{Output} & Saída correspondente ao bit 7 do registrador.\\
            \hline
            rp\_out & 2 bits & \textit{Output} & Saída correspondente aos bits 6 e 5 do registrador.\\
            \hline
            z\_out & 1 bit & \textit{Output} & Saída correspondente ao bit 2 do registrador.\\
            \hline\
            dc\_out & 1 bit & \textit{Output} & Saída correspondente ao bit 1 do registrador.\\
            \hline
            c\_out & 1 bit & \textit{Output} & Saída correspondente ao bit 0 do registrador.\\
            \hline
        \end{tabular}
    \end{center}
    \caption{Entradas e Saídas de status\_reg}
\end{table}

\subsection{Implementação}

O status\_reg foi implementado utilizando a linguagem VHDL.\\

O código na íntegra está abaixo:\\

\begin{lstlisting}[language=VHDL, caption={Código VHDL status\_reg}]
LIBRARY ieee;
USE ieee.std_logic_1164.all;
USE ieee.std_logic_unsigned.all;
USE ieee.numeric_std.all;

ENTITY status_reg IS
    PORT (
        -- Inputs
        nrst: IN STD_LOGIC;                             -- Reset
        clk_in: IN STD_LOGIC;                           -- Clock
        abus_in: IN STD_LOGIC_VECTOR(8 DOWNTO 0);       -- Enderecamento
        dbus_in: IN STD_LOGIC_VECTOR(7 DOWNTO 0);       -- Dados
        wr_en: IN STD_LOGIC;                            -- Enable escrita
        rd_en: IN STD_LOGIC;                            -- Enable leitura
        z_in: IN STD_LOGIC;                             -- Dados bit 2
        dc_in: IN STD_LOGIC;                            -- Dados bit 1
        c_in: IN STD_LOGIC;                             -- Dados bit 0
        z_wr_en: IN STD_LOGIC;                          -- Enable escrita bit 2
        dc_wr_en: IN STD_LOGIC;                         -- Enable escrita bit 1
        c_wr_en: IN STD_LOGIC;                          -- Enable escrita bit 0

        -- Outputs
        dbus_out: OUT STD_LOGIC_VECTOR(7 DOWNTO 0);     -- Dados
        irp_out: OUT STD_LOGIC;                         -- Dados bit 7
        rp_out: OUT STD_LOGIC_VECTOR(1 DOWNTO 0);       -- Dados bit 6 e 5
        z_out: OUT STD_LOGIC;                           -- Dados bit 2
        dc_out: OUT STD_LOGIC;                          -- Dados bit 1
        c_out: OUT STD_LOGIC                            -- Dados bit 0
    );  
END ENTITY;

ARCHITECTURE status_reg OF status_reg IS
    SIGNAL mem_reg: STD_LOGIC_VECTOR(7 downto 0);
BEGIN
    PROCESS(nrst, clk_in, mem_reg, wr_en, z_in, dc_in, c_in)
    BEGIN
        IF nrst = '0' THEN 
            mem_reg <= "00000000";
        ELSIF RISING_EDGE(clk_in) THEN
            IF wr_en = '1' AND abus_in(6 DOWNTO 0) = "0000011" THEN
                mem_reg <= dbus_in;
            END IF;
            IF z_wr_en = '1' THEN
                mem_reg(2) <= z_in;
            END IF;
            IF dc_wr_en = '1' THEN
                mem_reg(1) <= dc_in;
            END IF;
            IF c_wr_en = '1' THEN
                mem_reg(0) <= c_in;
            END IF;
        END IF;
    END PROCESS;
    
    dbus_out <= mem_reg WHEN rd_en = '1' AND abus_in(6 DOWNTO 0) = "0000011" ELSE "ZZZZZZZZ";
    irp_out <= mem_reg(7);
    rp_out <= mem_reg(6 DOWNTO 5);
    z_out <= mem_reg(2);
    dc_out <= mem_reg(1);
    c_out <= mem_reg(0);
END status_reg;
\end{lstlisting}

\subsection{Simulação}

Para testar nosso código VHDL e certificar-nos de que nosso circuito funciona de maneira esperada, simulamos alguns casos de testes utilizando o software Quatus II.\\

Os testes realizados foram os seguites:

\begin{enumerate}
    \item Escrita com enderaçamento incorreto (diferente de ``0bXX0000011).
    \begin{itemize}
        \item \textbf{Comportamento esperado:}
        \begin{itemize}
            \item dbus\_out em alta impedância;
            \item rp\_out = ``0b00''
            \item irp\_out = z\_out = dc\_out = c\_out = ``0b0''
        \end{itemize}
    \end{itemize}
    
    \item Leitura desabilitada e escrita habilitada;
    \begin{itemize}
        \item dbus\_in = ``0b01011000''.\
        \item z\_in = dc\_in = c\_in = 1;
        \item z\_wr\_en = dc\_wr\_en = c\_wr\_en = 0;
        \item \textbf{Comportamento esperado:}
        \begin{itemize}
            \item dbus\_out em alta impendância;
            \item rp\_out = ``10b''
            \item irp\_out = z\_out = dc\_out = c\_out = ``0b0''
        \end{itemize}
    \end{itemize}

    \item Leitura habilitada e escrita desabilitada.
    \begin{itemize}
        \item Valor salvo = ``0b01011000''.
        \item \textbf{Comportamento esperado:}
        \begin{itemize}
            \item dbus\_out = ``0b01011000'';
            \item rp\_out = ``10b''
            \item irp\_out = z\_out = dc\_out = c\_out = ``0b0''
        \end{itemize}
    \end{itemize}

    \item Leitura desabilitada e escrita habilitada;
    \begin{itemize}
        \item dbus\_in = ``0b10100000''.
        \item z\_in = dc\_in = c\_in = 1;
        \item z\_wr\_en = dc\_wr\_en = c\_wr\_en = 1;
        \item \textbf{Comportamento esperado:}
        \begin{itemize}
            \item dbus\_out em alta impendância;
            \item rp\_out = ``01b''
            \item irp\_out = z\_out = dc\_out = c\_out = ``0b1''
        \end{itemize}
    \end{itemize}

    \item Leitura habilitada e escrita desabilitada.
    \begin{itemize}
        \item Valor salvo = ``0b10100111''.
        \item \textbf{Comportamento esperado:}
        \begin{itemize}
            \item dbus\_out = ``0b10100111'';
            \item rp\_out = ``10b''
            \item irp\_out = z\_out = dc\_out = c\_out = ``0b1''
        \end{itemize}
    \end{itemize}

    \item \textit{Reset} com leitura habilitada.
    \begin{itemize}
        \item \textbf{Comportamento esperado:}
        \begin{itemize}
            \item dbus\_out = ``0b00000000'';
            \item rp\_out = ``00b''
            \item irp\_out = z\_out = dc\_out = c\_out = ``0b0''
        \end{itemize}
    \end{itemize}
\end{enumerate}

\begin{figure}[ht]
    \begin{center}
        \includegraphics[width=15cm]{images/status.png}
        \caption{Simulação status\_reg}
\end{center}
\end{figure}

\newpage

\section{stack}

O bloco \textbf{stack} é um conjunto de 8 registradores de 13 bits. As operações de \textit{push} e \textit{pop} adicionam e removem dados da pilha, respectivamente.\\

As entradas e saídas deste bloco estão descritas na tabela abaixo:\\

\begin{table}[ht]
    \begin{center}
        \begin{tabular}{|c|c|c|c|}
            \hline
            Nome & Tamanho & Tipo & Descrição\\
            \hline
            nrst & 1 bit & \textit{Input} & Entrada de \textit{reset} assíncrono.\\
            \hline
            clk\_in & 1 bit & \textit{Input} & Entrada de \textit{clock}.\\
            \hline
            stack\_in & 13 bit & \textit{Input} & Entrada de dados para a pilha.\\
            \hline
            stack\_push & 1 bit & \textit{Input} & Entrada de habilitação para colocar valores na pilha.\\
            \hline
            stack\_pop & 1 bit & \textit{Input} & Entrada de habilitação para retirar valores da pilha.\\
            \hline
            stack\_out & 13 bits & \textit{Output} & Saída correspondente à primeira posição da pilha.\\
            \hline
        \end{tabular}
    \end{center}
    \caption{Entradas e Saídas do bloco stack}
\end{table}

\subsection{Implementação}

O bloco stack foi implementado utilizando a linguagem VHDL.\\

O código na íntegra está abaixo:\\

\begin{lstlisting}[language=VHDL, caption={Código VHDL stack}]
LIBRARY ieee;
USE ieee.std_logic_1164.all;
USE ieee.std_logic_unsigned.all;
USE ieee.numeric_std.all;

ENTITY stack IS
    PORT (
        -- Inputs
        nrst: IN STD_LOGIC;                             -- Reset
        clk_in: IN STD_LOGIC;                           -- Clock
        stack_in: IN STD_LOGIC_VECTOR(12 DOWNTO 0);     -- Dados
        stack_push: IN STD_LOGIC;                       -- Enable push op
        stack_pop: IN STD_LOGIC;                        -- Enable pop op
        
        -- Outputs
        stack_out: OUT STD_LOGIC_VECTOR(12 DOWNTO 0)    -- Stack output
    );
END ENTITY;

ARCHITECTURE stack OF stack IS
    SIGNAL mem_reg1, mem_reg2, mem_reg3, mem_reg4, mem_reg5, mem_reg6, mem_reg7, mem_reg8 : STD_LOGIC_VECTOR(12 DOWNTO 0);
BEGIN
    PROCESS(nrst, clk_in, stack_push, stack_pop)
    BEGIN
        IF nrst = '0' THEN
            mem_reg1 <= "0000000000000";
            mem_reg2 <= "0000000000000";
            mem_reg3 <= "0000000000000";
            mem_reg4 <= "0000000000000";
            mem_reg5 <= "0000000000000";
            mem_reg6 <= "0000000000000";
            mem_reg7 <= "0000000000000";
            mem_reg8 <= "0000000000000";
        ELSIF RISING_EDGE(clk_in) THEN
			stack_out <= "0000000000000";
            IF stack_pop = '1' THEN
                stack_out <= mem_reg1;
                mem_reg1 <= mem_reg2;
                mem_reg2 <= mem_reg3;
                mem_reg3 <= mem_reg4;
                mem_reg4 <= mem_reg5;
                mem_reg5 <= mem_reg6;
                mem_reg6 <= mem_reg7;
                mem_reg7 <= mem_reg8;
                mem_reg8 <= "0000000000000";
            ELSIF stack_push = '1' THEN
                mem_reg8 <= mem_reg7;
                mem_reg7 <= mem_reg6;
                mem_reg6 <= mem_reg5;
                mem_reg5 <= mem_reg4;
                mem_reg4 <= mem_reg3;
                mem_reg3 <= mem_reg2;
                mem_reg2 <= mem_reg1;
                mem_reg1 <= stack_in; 
            END IF;
        END IF;
    END PROCESS;
END stack;
\end{lstlisting}

\subsection{Simulação}

Para testar nosso código VHDL e certificar-nos de que nosso circuito funciona de maneira esperada, simulamos alguns casos de testes utilizando o software Quatus II.\\

Os testes realizados foram os seguites:

\begin{enumerate}
    \item \textit{Push} até pilha cheia.
    \begin{itemize}
        \item stack\_in: Sequência de ``0'' à ``7'';
        \item stack\_push = ``1'';
        \item stack\_pop = ``0'';
        \item \textbf{Comportamento esperado:}
        \begin{itemize}
            \item stack\_out = ``0'';
        \end{itemize}
    \end{itemize}

    \item \textit{Pop} até pilha vazia.
    \begin{itemize}
        \item stack\_push = ``0'';
        \item stack\_pop = ``1'';
        \item \textbf{Comportamento esperado:}
        \begin{itemize}
            \item stack\_out = Sequência de ``7'' à ``0'';
        \end{itemize}
    \end{itemize}

    \item \textit{Push} até \textit{Stack Overvlow}.
    \begin{itemize}
        \item stack\_in: Sequência de ``0'' à ``9'';
        \item stack\_push = ``1'';
        \item stack\_pop = ``0'';
        \item \textbf{Comportamento esperado:}
        \begin{itemize}
            \item stack\_out = ``0'';
        \end{itemize}
    \end{itemize}

    \item \textit{Pop} até \textit{Stack Underflow}.
    \begin{itemize}
        \item stack: Sequência de ``9'' à ``2'';
        \item stack\_push = ``0'';
        \item stack\_pop = ``1'';
        \item \textbf{Comportamento esperado:}
        \begin{itemize}
            \item stack\_out = Sequência de ``9'' à ``2'', depois, ``0'';
        \end{itemize}
    \end{itemize}

    \item \textit{Push} e \textit{Pop} simultâneo.
    \begin{itemize}
        \item stack: ``1'';
        \item stack\_push = ``1'';
        \item stack\_pop = ``1'';
        \item \textbf{Comportamento esperado:}
        \begin{itemize}
            \item Preferência do \textit{pop}.
            \item stack\_out = ``1'', depois, ``0'';
        \end{itemize}
    \end{itemize}
\end{enumerate}

\begin{figure}[ht]
    \begin{center}
        \includegraphics[width=15cm]{images/stack.png}
        \caption{Simulação bloco stack}
\end{center}
\end{figure}

\newpage

\section{pc\_reg}

O bloco pc\_reg controla o registrados pc de um processador. Possui operações para incremento de endereço, \textit{load}, \textit{reset} e escrita, bem como operações de \textit{push} e \textit{pop} adicionam e removem o valor corrente de pc à uma pilha pilha.\\

As entradas e saídas deste bloco estão descritas na tabela abaixo:\\

\begin{table}[ht]
    \begin{center}
        \begin{tabular}{|c|c|c|c|}
            \hline
            Nome & Tamanho & Tipo & Descrição\\
            \hline
            nrst & 1 bit & \textit{Input} & Entrada de \textit{reset} assíncrono.\\
            \hline
            clk\_in & 1 bit & \textit{Input} & Entrada de \textit{clock}.\\
            \hline
            aadr\_in & 11 bits & \textit{Input} & Entrada de dados para carga no registrador PC.\\
            \hline
            abus\_in & 9 bits & \textit{Input} & Entrada de endereçamento para PCL e para o registrador PCLATH.\\
            \hline
            dbus\_in & 8 bits & \textit{Input} & Entrada de dados para escrita em PCL e PCLATH.\\
            \hline
            inc\_pc & 1 bit & \textit{Input} & Entrada de habilitação para incremento.\\
            \hline
            load\_pc & 1 bit & \textit{Input} & Entrada de habilitação para carga.\\
            \hline
            wr\_en & 1 bit & \textit{Input} & Entrada de habilitação para escrita nos registradores.\\
            \hline
            rd\_en & 1 bit & \textit{Input} & Entrada de habilitação para leitura dos registradores.\\
            \hline
            stack\_push & 1 bit & \textit{Input} & Entrada de habilitação para colocar valores na pilha.\\
            \hline
            stack\_pop & 1 bit & \textit{Input} & Entrada de habilitação para retirar valores da pilha.\\
            \hline
            nextpc\_out & 13 bits & \textit{Output} & Saída do valor a ser carregado no contador.\\
            \hline
            dbus\_out & 8 bits & \textit{Output} & Saída de dados lidos com endereçamento por abus\_in.\\
            \hline
        \end{tabular}
    \end{center}
    \caption{Entradas e Saídas do bloco pc\_reg}
\end{table}

\subsection{Implementação}

O bloco pc\_reg foi implementado utilizando a linguagem VHDL.\\

O código na íntegra está abaixo:\\

\begin{lstlisting}[language=VHDL, caption={Código VHDL pc\_reg}]
LIBRARY ieee;
USE ieee.std_logic_1164.all;
USE ieee.std_logic_unsigned.all;
USE ieee.numeric_std.all;

ENTITY pc_reg IS
    PORT (
        -- Inputs
        nrst: IN STD_LOGIC;                             -- Reset
        clk_in: IN STD_LOGIC;                           -- Clock
        addr_in: IN STD_LOGIC_VECTOR(10 DOWNTO 0);      -- Dados
        abus_in: IN STD_LOGIC_VECTOR(8 DOWNTO 0);       -- Enderecamento PCL e PCLATH
        dbus_in: IN STD_LOGIC_VECTOR(7 DOWNTO 0);       -- Dados PCL e PCLATH
        inc_pc: IN STD_LOGIC;                           -- Enable incremento.
        load_pc: IN STD_LOGIC;                          -- Enable carga.
        wr_en: IN STD_LOGIC;                            -- Enable escrita.
        rd_en: IN STD_LOGIC;                            -- Enable leitura.
        stack_push: IN STD_LOGIC;                       -- Enable push op
        stack_pop: IN STD_LOGIC;                        -- Enable pop op

        -- Outputs
        nextpc_out: OUT STD_LOGIC_VECTOR(12 DOWNTO 0);  -- Contador
        dbus_out: OUT STD_LOGIC_VECTOR(7 DOWNTO 0)      -- Dados
    );
END ENTITY;

ARCHITECTURE pc_reg OF pc_reg IS
    SIGNAL stack_reg1, stack_reg2, stack_reg3, stack_reg4, stack_reg5, stack_reg6, stack_reg7, stack_reg8 : STD_LOGIC_VECTOR(12 DOWNTO 0);
    SIGNAL stack_popped: STD_LOGIC_VECTOR(12 DOWNTO 0);
    SIGNAL pc: STD_LOGIC_VECTOR(12 DOWNTO 0);
    SIGNAL lath_pc: STD_LOGIC_VECTOR(7 DOWNTO 0);
    SIGNAL nextpc: STD_LOGIC_VECTOR(12 DOWNTO 0);
BEGIN

    -- Stack
    PROCESS(nrst, clk_in, stack_push, stack_pop, stack_popped)
    BEGIN
        IF nrst = '0' THEN
            stack_reg1 <= "0000000000000";
            stack_reg2 <= "0000000000000";
            stack_reg3 <= "0000000000000";
            stack_reg4 <= "0000000000000";
            stack_reg5 <= "0000000000000";
            stack_reg6 <= "0000000000000";
            stack_reg7 <= "0000000000000";
            stack_reg8 <= "0000000000000";
        ELSIF RISING_EDGE(clk_in) THEN
            IF stack_pop = '1' THEN
                stack_popped <= stack_reg1;
                stack_reg1 <= stack_reg2;
                stack_reg2 <= stack_reg3;
                stack_reg3 <= stack_reg4;
                stack_reg4 <= stack_reg5;
                stack_reg5 <= stack_reg6;
                stack_reg6 <= stack_reg7;
                stack_reg7 <= stack_reg8;
                stack_reg8 <= "0000000000000";
            ELSIF stack_push = '1' THEN
                stack_reg8 <= stack_reg7;
                stack_reg7 <= stack_reg6;
                stack_reg6 <= stack_reg5;
                stack_reg5 <= stack_reg4;
                stack_reg4 <= stack_reg3;
                stack_reg3 <= stack_reg2;
                stack_reg2 <= stack_reg1;
                stack_reg1 <= pc; 
            END IF;
        END IF;
    END PROCESS;

    -- logica combinacional para nextpc
    PROCESS(stack_pop, inc_pc, load_pc, wr_en, abus_in, pc, addr_in, lath_pc, stack_popped, dbus_in)
    BEGIN
        IF stack_pop = '1' THEN
            nextpc <= stack_popped;
        ELSIF inc_pc = '1' THEN
            nextpc <= pc + 1;
        ELSIF load_pc = '1' THEN
            nextpc(10 DOWNTO 0) <= addr_in;
            nextpc(12 DOWNTO 11) <= lath_pc(4 DOWNTO 3);
        ELSIF wr_en = '1' AND abus_in(6 DOWNTO 0) = "0000010" THEN
            nextpc <= lath_pc(4 DOWNTO 0) & dbus_in;
        ELSE
            nextpc <= pc;
        END IF;
        
        nextpc_out <= nextpc;
    END PROCESS;

    -- logica sequencial para PC_reg
    PROCESS(clk_in, nrst, pc, nextpc)
    BEGIN
        IF RISING_EDGE(clk_in) THEN
            pc <= nextpc;
        END IF;

        IF nrst = '0' THEN
            pc <= "0000000000000";
        END IF;
    END PROCESS;

    -- logica sequencial para PCLATH
    PROCESS(clk_in, nrst, wr_en, abus_in)
    BEGIN
        IF RISING_EDGE(clk_in) THEN
            IF wr_en = '1' AND abus_in(6 DOWNTO 0) = "0001010" THEN
                lath_pc <= dbus_in;
            END IF;
        END IF;
        
        IF nrst = '0' THEN
            lath_pc <= "00000000";
        END IF;
    END PROCESS;

    -- logica combinacional para dbus_out
    PROCESS(clk_in, rd_en, abus_in, lath_pc, pc)
    BEGIN
        IF abus_in(6 DOWNTO 0) = "0001010" AND rd_en = '1' THEN
            dbus_out <= lath_pc(7 DOWNTO 0);
        ELSIF abus_in(6 DOWNTO 0) = "0000010" AND rd_en = '1' THEN
            dbus_out <= pc(7 DOWNTO 0);
        ELSE
            dbus_out <= "ZZZZZZZZ";
        END IF;
    END PROCESS;

END pc_reg;
\end{lstlisting}

\subsection{Simulação}

Para testar nosso código VHDL e certificar-nos de que nosso circuito funciona de maneira esperada, simulamos alguns casos de testes utilizando o software Quatus II.\\

Os testes realizados foram os seguites:

\begin{enumerate}
    \item Incremento de PC (4x).
    \begin{itemize}
        \item inc\_pc: \textit{High} (4x);
        \item \textbf{Comportamento esperado:}
        \begin{itemize}
            \item nextpc\_out: Sequência de ``0x1'' à ``0x4'';
            \item dbus\_out: Alta impedância;
        \end{itemize}
    \end{itemize}

    \item \textit{Reset} do PC.
    \begin{itemize}
        \item nrst: \textit{High};
        \item \textbf{Comportamento esperado:}
        \begin{itemize}
            \item nextpc\_out: ``0x00'';
            \item dbus\_out: Alta impedância;
        \end{itemize}
    \end{itemize}

    \item Empilhamento e Desempilhamento de PC (``0x0'' e ``0x2'').
    \begin{itemize}
        \item stack\_push: \textit{High} (2x) e \textit{low} (2x);
        \item stack\_pop: \textit{Low} (2x) e \textit{high} (2x);
        \item \textbf{Comportamento esperado:}
        \begin{itemize}
            \item nextpc\_out: ``0x02'' e ``0x00'', nesta ordem;
            \item dbus\_out: Alta impedância;
        \end{itemize}
    \end{itemize}

    \item Escrita e leitura de PCL (``0x0'' e ``0x2'').
    \begin{itemize}
        \item abus\_in: ``0x002'';
        \item dbus\_in: ``0x0A''
        \item wr\_en: \textit{High} na escrita. \textit{low} na leitura;
        \item rd\_en: \textit{Low} na escrita. \textit{high} na leitura;
        \item \textbf{Comportamento esperado:}
        \begin{itemize}
            \item nextpc\_out: ``0x?0A'';
            \item dbus\_out: ``0x0A'';
        \end{itemize}
    \end{itemize}
    
    \item Carregamento de endereço.
    \begin{itemize}
        \item addr\_in: ``0x3AB'';
        \item load\_pc: \textit{High};
        \item \textbf{Comportamento esperado:}
        \begin{itemize}
            \item nextpc\_out: ``0x3AB'';
            \item dbus\_out: Alta impedância;
        \end{itemize}
    \end{itemize}

    \item Escrita e leitura de PCLATH (``0x0'' e ``0x2'').
    \begin{itemize}
        \item abus\_in: ``0x00A'';
        \item dbus\_in: ``0x18'';
        \item wr\_en: \textit{High} na escrita. \textit{low} na leitura;
        \item rd\_en: \textit{Low} na escrita. \textit{high} na leitura;
        \item \textbf{Comportamento esperado:}
        \begin{itemize}
            \item nextpc\_out: ``0x?18'';
            \item dbus\_out: ``0x0A'';
        \end{itemize}
    \end{itemize}
\end{enumerate}

\begin{figure}[ht]
    \begin{center}
        \includegraphics[width=15cm]{images/pc_reg.png}
        \caption{Simulação bloco pc\_reg}
\end{center}
\end{figure}

\section{Conclusão}

A implementação de registradores através de liguagens de descrição de hardware, como a VHDL, são maneiras poderosas de construir circuitos computacionais de maneira menos complexa e mais rápida.

Apesar de parecerem ``simples'' à primeira vista, sua implementação possuem nuancias que exigem a atenção do engenheiro, e testes exaustivos para certificar que funcionam como deveriam.

Neste trabalho prático, aprendemos como construir registradores e pilhas para auxiliar a contrução de nossos circuitos computacionais, utilizando conhecimentos da prática anterior e novas técnicas, como a utilização do ``PROCESS'', que permite a execução de trechos sequenciais no nosso circuito, e a tuilização do \textit{clock},  que permite a sincronização de rotinas no nosso cicuito.

Com os circuitos implementados, estamos mais confiantes nas nossas capacidades, e estamos um passo mais perto de implementar circuitos mais complexos como controladores e processadores.

\end{document}