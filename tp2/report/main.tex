\documentclass{article}

% Packages
\usepackage{geometry}
\usepackage{graphicx}
\usepackage{lipsum}
\usepackage{listings}
\usepackage[portuguese]{babel}

% -- Defining colors:
\usepackage[dvipsnames]{xcolor}
\definecolor{codegreen}{rgb}{0,0.6,0}
\definecolor{codegray}{rgb}{0.5,0.5,0.5}
\definecolor{codepurple}{rgb}{0.58,0,0.82}
\definecolor{backcolour}{rgb}{0.95,0.95,0.92}

% Definig a custom style:
\lstdefinestyle{mystyle}{
    backgroundcolor=\color{backcolour},   
    commentstyle=\color{codepurple},
    keywordstyle=\color{NavyBlue},
    numberstyle=\tiny\color{codegray},
    stringstyle=\color{codepurple},
    basicstyle=\ttfamily\footnotesize\bfseries,
    breakatwhitespace=false,         
    breaklines=true,                                   
    keepspaces=true,                 
    numbers=left,                    
    numbersep=5pt,                  
    showspaces=false,                
    showstringspaces=false,
    showtabs=false,                  
    tabsize=2,
    captionpos=b
}

% -- Setting up the custom style:
\lstset{style=mystyle}

% Page setup
\geometry{a4paper, margin=2cm}

\begin{document}

% cover
\thispagestyle{empty} % Remove page number from first page

\begin{figure}[t]
    \includegraphics[width=3cm]{images/logo-puc-minas.png}
    \hspace{0.02\textwidth}
    \vline%
    \hspace{0.04\textwidth}
    \includegraphics[width=3cm]{images/logo-icei.jpeg}
\end{figure}

\hrulefill%
\vspace{\baselineskip}

\Large\noindent
\textbf{Pontifícia Universidade Católica de Minas Gerais} \\
\textbf{Instituto de Ciências Exatas e Informática} \\
\textbf{Departamento de Engenharia de Computação}

\begin{center}
    \vfill
    \Huge\textbf{Relatório do Laboratório 5} \\
    \vspace{0.5cm}
    \Large\textbf{Fonte Regulada com Diodo Zener} \\
    \vspace{1cm}
    \large \textbf{Professor}: Bruno Luiz Dias Alves de Castro \\
    \vspace{0.5cm}
    \large Bruno Luiz Dias Alves de Castro \\
    \large Bruno Luiz Dias Alves de Castro \\
    \large Bruno Luiz Dias Alves de Castro \\
    \vfill
    \large Belo Horizonte \\ Campus Coração Eucarístico \\
    \vspace{\baselineskip}
    \large \today
\end{center}

% table of contents
\newpage
\thispagestyle{empty}
\tableofcontents

% body
\newpage
\large % document text size

\section{Introdução}

Durante as aulas da disciplina de Sistemas Reconfiguráveis, fomos introduzidos à linguagem VHDL. VHDL (\textbf{V}HSIC \textbf{H}ardware \textbf{D}escription \textbf{L}anguage) é uma linguagem de descrição de hardware. Com ela, podemos montar circuitos lógicos de maneira totalmente textual, o que garante à linguagem uma grande vantagem ante à soluções visuais.

\subsection{Objetivos}

\subsection{Simulação via Quartus II}

Nessa etapa realizamos testes no software Quartus II da altera.

\subsubsection{Bloco w\_reg}

Nesta imagem é realizado 3 testes para verificar a funcionalidade do registrador, nos primeiros 60ns é alterado os bits da entrada de dados (d\_in) para nivel lógico alto, o bit de reset (nrst) que é ativo em baixa, é desativado, ou seja, nível lógico alto e o bit de ativação (wr\_en) é colocoado em nível lógico alto após 10ns. Assim é possível verificar a mudança na saída (w\_out) com um tempo de delay de 6ns. No segundo teste a partir de 60ns até 140ns é resetado os bits da memória do registrador colocando reset em nível lógico zero, o resultado é propagada para a saída após o tempo de delay de aproximadamente 6ns. No terceiro teste foi verificados se o bit de ativação de escrita está funcionando corretamente, portanto com o bit 6 da saída em nível lógico alto esse valor será escrito apenas no tempo 160ns quando é colocado a porta de ativação do registrador em nível lógico alto e o registrador é escrito.

\begin{figure}[ht]
\begin{center}
    \includegraphics[width=15cm]{images/w_reg.png}
    \caption{Simulação bloco w\_reg}
\end{center}
\end{figure}

\section{fsr\_reg}

O registrador FSR é um registrador semelhante ao implemntado anteriormente. A principal diferença entre os dois está na presença de um sistema de endereçamento, e de duas entradas binárias independentes para habilitação da escrita e da leitura. Os requisitos são descritos na tabela abaixo.

\begin{center}
\begin{tabular}{|c|c|c|c|}
        \hline
        Nome & Tamanho & Tipo & Descrição\\
        \hline
        nrst & 1 bit & \textit{Input} & Entrada de \textit{reset} assíncrono.\\
        \hline
        clk\_in & 1 bit & \textit{Input} & Entrada de \textit{clock}.\\
        \hline
        abus\_in & 9 bit & \textit{Input} & Entrada de enderençamento\\
        \hline
        dbus\_in & 8 bits & \textit{Input} & Entrada de dados para escrita\\
        \hline
        wr\_sel & 1 bit & \textit{Input} & Entrada de habilitação de escrita\\
        \hline
        rd\_sel & 1 bit & \textit{Input} & Entrada de habilitação de leitura\\
        \hline
        dbus\_out & 8 bits & \textit{Output} & Saída de dadas hailitada por rd\_en\\
        \hline
        fsr\_out & 8 bits & \textit{Output} & Saída de dadas dados sempre ativa\\
        \hline
\end{tabular}
\end{center}

\subsection{Implementação}

O registrador fsr\_reg foi implementado utilizando a linguagem VHDL.\\

O código na íntegra está abaixo:\\

\begin{lstlisting}[language=VHDL, caption={Código VHDL fsr\_reg}]
LIBRARY ieee;
USE ieee.std_logic_1164.all;
USE ieee.std_logic_unsigned.all;
USE ieee.numeric_std.all;

ENTITY fsr_reg IS
    PORT (
        -- Inputs
        nrst : IN STD_LOGIC;                            -- Reset
        clk_in: IN STD_LOGIC;                           -- Clock
        abus_in: IN STD_LOGIC_VECTOR(8 DOWNTO 0);       -- Enderecamento
        dbus_in: IN STD_LOGIC_VECTOR(7 DOWNTO 0);       -- Dados
        wr_en : IN STD_LOGIC;                           -- Enable escrita
        rd_en : IN STD_LOGIC;                           -- Enable leitura

        -- Outputs
        dbus_out : OUT STD_LOGIC_VECTOR(7 DOWNTO 0);    -- Dados
        fsr_out : OUT STD_LOGIC_VECTOR(7 DOWNTO 0)      -- Registrador
    );
END ENTITY;

ARCHITECTURE fsr_reg OF fsr_reg IS
    SIGNAL mem_reg: STD_LOGIC_VECTOR(7 DOWNTO 0);
BEGIN
    PROCESS (nrst, clk_in, mem_reg, abus_in, dbus_in)
    BEGIN
        IF nrst = '0' THEN
            mem_reg <= "00000000";
        ELSIF abus_in(6 DOWNTO 0) = "0000100" THEN
            IF RISING_EDGE(clk_in) THEN
                IF wr_en = '1' THEN
                    mem_reg <= dbus_in;
                END IF;
            END IF;
        END IF;
    END PROCESS;

    dbus_out <= mem_reg WHEN rd_en = '1' ELSE "ZZZZZZZZ";
    fsr_out <= mem_reg;
END fsr_reg;    
\end{lstlisting}

\subsection{Simulação}

Para testar nosso código VHDL e certificar-nos de que nosso circuito funciona de maneira esperada, simulamos alguns casos de testes utilizando o software Quatus II.\\

Os testes realizados foram os seguites:

\begin{enumerate}
    \item Escrita com enderaçamento incorreto (diferente de XX0000100).
    \begin{itemize}
        \item \textbf{Comportamento esperado:}
        \begin{itemize}
            \item dbus\_out em alta impedância;
            \item fsr\_out sem alteração;
        \end{itemize}
    \end{itemize}
    
    \item Leitura habilitada e escrita desabilitada.
    \begin{itemize}
        \item \textbf{Comportamento esperado:}
        \begin{itemize}
            \item dbus\_out = frs\_out = último valor escrito;
        \end{itemize}
    \end{itemize}

    \item Leitura desabilitada e escrita habilitada.
    \begin{itemize}
        \item \textbf{Comportamento esperado:}
        \begin{itemize}
            \item dbus\_out em alta impendância;
            \item frs\_out = dbus\_in;
        \end{itemize}
    \end{itemize}

    \item \textit{Reset} com leitura habilitada.
    \begin{itemize}
        \item \textbf{Comportamento esperado:}
        \begin{itemize}
            \item dbus\_out = frs\_out = ``0b00000000'';
        \end{itemize}
    \end{itemize}
\end{enumerate}

\begin{figure}[ht]
    \begin{center}
        \includegraphics[width=15cm]{images/fsr_reg.png}
        \caption{Simulação fsr\_reg}
\end{center}
\end{figure}

\section{Conclusão}
Com estes dois projetos simples, tivemos um excelente primeiro contado com a linguagem VHDL, bem como à programação concorrente e desenvolvimento de circuitos FPGA. Os dois circuitos implementados (Multiplexador de Endereçamento e Unidade Lógica Aritimética) são blocos de construção chave para a maior parte dos circuitos complexos, e serão de suma importância não só para os demais trabalhos práticos que realizaremos ao longo do semestre, mas para nosso desenvolvimento acadêmico e profissional.

\end{document}
