\documentclass{article}

% Packages
\usepackage{geometry}
\usepackage{graphicx}
\usepackage{lipsum}
\usepackage[portuguese]{babel}

% Page setup
\geometry{a4paper, margin=2cm}

\begin{document}

\thispagestyle{empty} % Remove page number from first page

\begin{figure}[t]
    \includegraphics[width=3cm]{images/logo-puc-minas.png}
    \hspace{0.02\textwidth}
    \vline%
    \hspace{0.04\textwidth}
    \includegraphics[width=3cm]{images/logo-icei.jpeg}
\end{figure}

\hrulefill%
\vspace{\baselineskip}

\Large\noindent
\textbf{Pontifícia Universidade Católica de Minas Gerais} \\
\textbf{Instituto de Ciências Exatas e Informática} \\
\textbf{Departamento de Engenharia de Computação}

\begin{center}
    \vfill
    \Huge\textbf{Relatório do Laboratório 5} \\
    \vspace{0.5cm}
    \Large\textbf{Fonte Regulada com Diodo Zener} \\
    \vspace{1cm}
    \large \textbf{Professor}: Bruno Luiz Dias Alves de Castro \\
    \vspace{0.5cm}
    \large Bruno Luiz Dias Alves de Castro \\
    \large Bruno Luiz Dias Alves de Castro \\
    \large Bruno Luiz Dias Alves de Castro \\
    \vfill
    \large Belo Horizonte \\ Campus Coração Eucarístico \\
    \vspace{\baselineskip}
    \large \today
\end{center}

\newpage
\thispagestyle{empty}
\tableofcontents
\newpage

\large

\section{Introdução}
Durante as aulas da disciplina de Sistemas Reconfiguráveis, fomos introduzidos à linguagem VHDL. VHDL (\textbf{V}HSIC \textbf{H}ardware \textbf{D}escription \textbf{L}anguage) é uma linguage de descrição de hardware. Com ela, podemos montar circuitos lógicos de maneira totalmente textual, o que da à linguagem um gigante vantagem ante à soluções visuais.

\subsection{TP1}
Como primeiro trabalho prático, são propostas montagens de dois circuitos utilizando a linguagem VHDL para a placa de sensolvimento Altera:

\begin{itemize}
    \item Um addr\_mux (Secção 2)
    \item Uma ULA (Unidade Lógica Aritimética) (Secção 3)
\end{itemize}

\subsubsection{Objetivos}

Entre os objetivos que temos com o desenvolvimento deste tarbalho prático podemos listar:

\begin{itemize}
    \item Aprender conceitos básicos da linguagem VHDL;
    \item Implementar utilizando programação concorrente os dois circuitos propostos.
    \item Compilar os circuitos e testar os resultados na placa de desenvolvimento Altera.
\end{itemize}

\section{ADDR\_MUX}
\lipsum[1]

\section{ULA (Unidade Lógica Aritimética)}
\lipsum[3]

\section{Conclusion}
\lipsum[4]

\end{document}