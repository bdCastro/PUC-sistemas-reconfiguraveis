\documentclass{article}

% Packages
\usepackage{geometry}
\usepackage{graphicx}
\usepackage{lipsum}
\usepackage{listings}
\usepackage[portuguese]{babel}

% Page setup
\geometry{a4paper, margin=2cm}

\begin{document}

% cover
\thispagestyle{empty} % Remove page number from first page

\begin{figure}[t]
    \includegraphics[width=3cm]{images/logo-puc-minas.png}
    \hspace{0.02\textwidth}
    \vline%
    \hspace{0.04\textwidth}
    \includegraphics[width=3cm]{images/logo-icei.jpeg}
\end{figure}

\hrulefill%
\vspace{\baselineskip}

\Large\noindent
\textbf{Pontifícia Universidade Católica de Minas Gerais} \\
\textbf{Instituto de Ciências Exatas e Informática} \\
\textbf{Departamento de Engenharia de Computação}

\begin{center}
    \vfill
    \Huge\textbf{Relatório do Laboratório 5} \\
    \vspace{0.5cm}
    \Large\textbf{Fonte Regulada com Diodo Zener} \\
    \vspace{1cm}
    \large \textbf{Professor}: Bruno Luiz Dias Alves de Castro \\
    \vspace{0.5cm}
    \large Bruno Luiz Dias Alves de Castro \\
    \large Bruno Luiz Dias Alves de Castro \\
    \large Bruno Luiz Dias Alves de Castro \\
    \vfill
    \large Belo Horizonte \\ Campus Coração Eucarístico \\
    \vspace{\baselineskip}
    \large \today
\end{center}

% table of contents
\newpage
\thispagestyle{empty}
\tableofcontents

% body
\newpage
\large % document text size

\section{Introdução}

Durante as aulas da disciplina de Sistemas Reconfiguráveis, fomos introduzidos à linguagem VHDL. VHDL (\textbf{V}HSIC \textbf{H}ardware \textbf{D}escription \textbf{L}anguage) é uma linguagem de descrição de hardware. Com ela, podemos montar circuitos lógicos de maneira totalmente textual, o que garante à linguagem uma grande vantagem ante à soluções visuais.

\subsection{TP1}

Como primeiro trabalho prático (TP1), são propostas as montagens de dois circuitos utilizando a linguagem VHDL para a placa de desenvolvimento Altera:

\begin{itemize}
    \item Um ADDR\_MUX (Multiplexador de Endereçamento) (Secção 2)
    \item Uma ULA (Unidade Lógica Aritimética) (Secção 3)
\end{itemize}

Ambos os circuitos devem ser desenvolvidos utilizando programação concorrente, ou seja, sem trechos sequênciais no código-fonte.

\subsubsection{Objetivos}

Entre os objetivos que temos com o desenvolvimento deste trabalho prático podemos listar:

\begin{itemize}
    \item Aprender conceitos básicos da linguagem VHDL;
    \item Implementar utilizando programação concorrente os dois circuitos propostos;
    \item Compilar os circuitos e testar os resultados na placa de desenvolvimento Altera;
\end{itemize}

\section{ADDR\_MUX}
O multiplexador de endereços é um MUX com uma saída de endereço (abus\_out), de 9 bits, que deve ser igual a concatenação da entrada de seleção de banco para endereçamento indireto (irp\_in), de 1 bit, com a entrada para endereçamento indireto (ind\_addr\_in), de 8 bits, quando todos os bits da entrada de endereçamento direto (dir\_addr\_in) forem iguais a 0. Caso contrário, a saída deve ser a concatenação da entrada de seleção de banco para endereçamento direto (rp\_in), de 2 bits, concatenada com a entrada para endereçamento direto (dir\_addr\_in) de 7 bits.

\subsection {Tabela Verdade}

A lógica do multiplexador é simples e pode ser representada por uma pequena tabela verdade.

\begin{center}
    \begin{tabular}{|c|c|}
        \hline
        dir\_addr\_in & abus\_out \\
        \hline
              0000000 & irp\_in \& ind\_addr\_in \\
        \hline
          maior que 0 & rp\_in \& dir\_addr\_in \\
        \hline
    \end{tabular}
\end{center}

\subsection {Implementação em VHDL}

Na implementação em VHDL foram declaradas 5 portas utilizando os tipos STD\_LOGIC para a entrada de 1 bit e STD\_LOGIC\_VECTOR para as entradas e saída de vários bits, esse tipo foi importado da biblioteca IEEE.

\begin{lstlisting}[language=VHDL]
	PORT (
		-- Input
		rp_in : IN STD_LOGIC_VECTOR(1 DOWNTO 0);
		dir_addr_in : IN STD_LOGIC_VECTOR(6 DOWNTO 0);
		irp_in : IN STD_LOGIC;
		ind_addr_in : IN STD_LOGIC_VECTOR(7 DOWNTO 0);
		-- Output
		abus_out : OUT STD_LOGIC_VECTOR(8 DOWNTO 0)
	);
\end{lstlisting}

\subsubsection{Arquitetura}

A lógica da arquitetura é bem simples e pode ser descrita com apenas uma linha de código, onde foi utilizado a estrutura WHEN...ELSE para descrever as relações entre as portas de entrada e a porta de saída.

\begin{lstlisting}[language=VHDL]
    abus_out <= irp_in & ind_addr_in WHEN dir_addr_in = "0000000"
                ELSE rp_in & dir_addr_in;
\end{lstlisting}

\section{ULA (Unidade Lógica Aritimética)}

A ULA (\textbf{U}nidade \textbf{L}ógica \textbf{A}ritimétrica) é um dos componentes mais básicos de um processador. Como o nome já indica, a ULA é responsável por todas as operações lógicas (como OR, AND e Shift) e aritiméticas (como soma e subtração) realizadas em nosso circuito.

De maneira simplificada, a ULA receberá um comando, composto por seletores de operação e bits, e operandos. Na saída temos o resultado da operação desejada.

Como uma ULA opera de maneira concorrente, todas as operações implementadas são ``executadas ao mesmo tempo''. Um Multiplexador é usado para selecionar a operação correta.

\subsection{ULA Proposta}

Neste trabalho prático, a ULA proposta deve possuir as seguintes funções:

\begin{center}
    \begin{tabular}{|c|c|c|c|c|c|}
        \hline
        Função & op\_code & Descrição & z\_out & c\_out & dc\_out\\
        \hline
        XOR & 0000 & XOR Lógico & 1, se res. = 0 & - & -\\
        \hline
        OR & 0001 & OR Lógico & 1, se res. = 0 & - & -\\
        \hline
        AND & 0010 & AND Lógico & 1, se res. = 0 & - & -\\
        \hline
        CLR & 0011 & Limpa & 1 & - & -\\ 
        \hline
        ADD & 0100 & Soma & 1, se res. = 0 & 1 se \textit{carry} & 1 se \textit{carry} no nibble\\
        \hline
        SUB & 0101 & Subtração & 1, se res. = 0 & 0 se \textit{borrow} & 0 se \textit{borrow} no nibble\\
        \hline
        INC & 0110 & Incremento & 1, se res. = 0 & - & - \\
        \hline
        DEC & 0111 & Decremento & 1, se res. = 0 & - & - \\
        \hline
        PASS\_A & 1000 & Passa 'A' & 1, se res. = 0 & - & - \\
        \hline
        PASS\_B & 1001 & Passa 'B' & 1, se res. = 0 & - & - \\
        \hline
        COM & 1010 & Complemento & 1, se res. = 0 & - & - \\
        \hline
        SWAP & 1011 & Permuta \textit{nibbles} & 1, se res. = 0 & - & - \\
        \hline
        BS & 1100 & bit\_sel = 1 & a\_in[bit\_sel] & - & - \\
        \hline
        BC & 1101 & bit\_sel = 0 & a\_in[bit\_sel] & - & - \\
        \hline
        RR & 1110 & Rotação p/ dir. & - &  a\_in[0] & - \\
        \hline
        RL & 1111 & Rotação p/ esq. & - &  a\_in[7] & - \\
        \hline
    \end{tabular}
\end{center}

O sinais de entrada e saída são os seguintes:

\begin{center}
\begin{tabular}{|c|c|c|c|}
    \hline
    Nome & Tamanho & Tipo & Descrição\\
    \hline
    a\_in & 8 bits & \textit{Input} & Entrada de dados A\\
    \hline
    b\_in & 8 bits & \textit{Input} & Entrada de dados B\\
    \hline
    c\_in & 1 bit & \textit{Input} & Entrada de \textit{carry}\\
    \hline
    op\_sel & 4 bits & \textit{Input} & Seletor de operação\\
    \hline
    bit\_sel & 3 bits & \textit{Input} & Seletor de bit\\
    \hline
    r\_out & 8 bits & \textit{Output} & Saída do resultado\\
    \hline
    c\_out & 1 bit & \textit{Output} & Saída de \textit{carry/borrow}\\
    \hline
    dc\_out & 1 bit & \textit{Output} & Saída de \textit{digit carry/borrow}\\
    \hline
    z\_out & 1 bit & \textit{Output} & Saída de zero\\
    \hline
\end{tabular}
\end{center}

\subsection{Implementação}

Nossa implementação, feita em VHDL, consiste na utilização de métodos e bibliotecas já implementadas, bem como lógica implementada por nós. Abaixo explicamos como foram implentadas cada uma das funções propostas.

\subsubsection{XOR, OR, AND e COM}

Esses métodos foram implementados usando as funções lógicas disponíveis nativamente na linguagem VHDL.

\begin{lstlisting}[language=VHDL]
aux <=
    a XOR b WHEN "0000",    -- XOR
    a OR b  WHEN "0001",    -- OR
    a AND b WHEN "0010",    -- AND

    [...]

    NOT a   WHEN "1010",    -- COM
\end{lstlisting}

\subsubsection{ADD, SUB, INC e DEC}

As funções de ADD (Adição), SUB (Subtração), INC (Incremento) e DEC (Decremento) foram implementadas usando operações aritiméticas já inclusas na linguagem. Para as funções de ADD e SUB, os operandos A e B são utilizados, e já para as operações de INC e DEC, apenas o operando A e a constante 1.

\begin{lstlisting}[language=VHDL]
aux <=  
    a + b   WHEN "0100",    -- ADD
    a - b   WHEN "0101",    -- SUB
    a + 1   WHEN "0110",    -- INC
    a - 1   WHEN "0111",    -- DEC
\end{lstlisting}

\subsubsection{PASS\_A e PASS\_B}

Talvez as operações mais simples, ambas não utilizam nenhuma lógica. As entradas A e B são apenas ``copiadas'' para a saída.

\begin{lstlisting}[language=VHDL]
aux <=  
    a   WHEN "1000",        -- PASS_A
    b   WHEN "1001",        -- PASS_B
\end{lstlisting}

\subsubsection{SWAP}

A operação de SWAP é feita invertendo os dois \textit{nibbles}.

\begin{lstlisting}[language=VHDL]
aux <= '0' & a(3 DOWNTO 0) & a(7 DOWNTO 4) WHEN "1011", -- SWAP
\end{lstlisting}

A concatenação com zero no início será explicada na implementação do c\_out.

\subsubsection{BS e BC}

As operações de BS (\textit{bit set}) e BC (\textit{bit clear}) implementam a seguinte lógica: para ``setarmos'' um bit para 1, podemos fazer uma operação OR da entrada com uma \textit{string} de zeros, mas com um único 1 na posição que desejamos que seja igual à 1. Isso garantirá que este bit será sempre 1, e os demais são copiados. Podemos obter essa \textit{string} efetuando um \textit{shift lógico} de 1 um número bit\_sel de casas.

A operação BC funciona de maneira semalhando, mas deve ser feita com um AND no lugar do OR, e a \textit{string} deve ser de 1s com um único zero na posição desejada. Para isso, fazemos mesma operação que anteriormente, mas invertemos o resultado.

\begin{lstlisting}[language=VHDL]
aux <=
    a OR
        STD_LOGIC_VECTOR(SHIFT_LEFT(TO_UNSIGNED(1, 8),
            TO_INTEGER(UNSIGNED(bit_sel))))
        WHEN "1100",    -- BS

    a AND
        NOT STD_LOGIC_VECTOR(SHIFT_LEFT(TO_UNSIGNED(1, 8),
            TO_INTEGER(UNSIGNED(bit_sel))))
        WHEN "1101",	-- BC
\end{lstlisting}

\subsubsection{RR e RL}

As operações de RR (\textit{Rotate Right}) e RL (\textit{Rotate Left}) foram implementadas selecionando os bits necessários, e concatenando à entrada c\_in.

\begin{lstlisting}[language=VHDL]
aux <=
    '0' & c_in & a(7 DOWNTO 1) WHEN "1110",     -- RR
    '0' & a(6 DOWNTO 0) & c_in WHEN "1111";     -- RL
\end{lstlisting}

\subsubsection{z\_out}

A saída z\_out é usada para indicar que o resultado é zero para a maioria das operações, com duas excessões: a de CLR (já que o resultado sempre é zero), e as operações BS e BC onde ele deve ser o bit da entrada A na posição bit\_sel.

Para isso, verificamos se a operação termina com a sequência "110" (utilizada por ambas as operações) e colocamos o bit correto na saída. Caso contrário, ela é 1 caso a saída seja zero.

\begin{lstlisting}[language=VHDL]
z_out <=
    a(TO_INTEGER(UNSIGNED(bit_sel)))
        WHEN op_sel(3 DOWNTO 1) = "110" -- BS and BC

    ELSE '1' WHEN aux(7 DOWNTO 0) = "00000000";
\end{lstlisting}

\subsubsection{c\_out}

Para idetificarmos um \textit{carry} (ou um \textit{borrow}), há várias rotas que podemos tomar. Poderiamos, por exemplo, implementar cirtuitos lógicos capazes de identificar que ele ocorreu. Mas, para simplificar a operação, podemos chegar no mesmo resultado simplesmente aumentando o número de bits dos nossos dois operandos em 1. Podemos então copiar o bit mais significativo para a a saída c\_out. Caso um \textit{carry} ou um \textit{borrow} ocorra, esse bit será um.

\begin{lstlisting}[language=VHDL]
    r_out <= aux(7 DOWNTO 0);
    c_out <= aux(8);
\end{lstlisting}

Para evitarmos que o $9^o$ bit seja copiado para a saída, usamos um buffer aux, e apenas os bits de 7 à 0 são copiados para a saída c\_out.

\subsubsection{dc\_out}

Para identificarmos um \textit{carry} ou \textit{borrow} em um \textit{nibble}, podemos fazer algo similar ao que fizemos c\_out. Criamos um auxiliar com 5 bits, e verificamos o copiamos para a saída, após executada a operação necessária.

\begin{lstlisting}[language=VHDL]
aux_nibble <=
    ('0' & a(3 DOWNTO 0)) +
        ('0' + b(3 DOWNTO 0)) WHEN op_sel(0) = '0' 
	ELSE ('0' & a(3 DOWNTO 0)) - ('0' & b(3 DOWNTO 0));

dc_out <= aux_nibble(4);
\end{lstlisting}

\subsection{Simulação e Testes}
Nesta fase do projeto, precisamos ter certeza de que tudo o que fizemos se comportava da maneira esperada. Para isso, efetuamos dois tipos de validações:

\begin{itemize}
    \item Simulação via Quartus II.
    \item Teste com a placa de desenvolvimento Altera.
\end{itemize}

\subsubsection{Simulação via Quartus II}
A primeira maneira de testar nosso circuito, foi pelo próprio software que utilizamos para toda a implementação: O Quartus II. Dentro dele temos uma função de simulação, Com ela, podemos simular as entradas como ``ondas'' no tempo, e analizar se obtemos as saídas esperadas.

Nessa fase, obtivemos as saídas corretas para todos os casos de testes, e não foi preciso fazer nenhuma alteração. Com isso, avançamos para a próxima fase.

\begin{figure}
\begin{center}
    \includegraphics[width=8cm]{images/xor.png}
    \caption{Simulação XOR}
\end{center}
\end{figure}

\begin{figure}
\begin{center}
    \includegraphics[width=8cm]{images/add.png}
    \caption{Simulação ADD\\}
\end{center}
\end{figure}

\begin{figure}
\begin{center}
    \includegraphics[width=8cm]{images/bc.png}
    \caption{Simulação BC}
\end{center}
\end{figure}

\subsubsection{Teste com a placa de desenvolvimento Altera}
Utilizando o mesmo software Quartus, compilamos o nosso projeto e subimos para a placa de testes Altera.

Nesta fase, tivemos que fazer uma pequena alteração. Apesar da placa ofereçer um grande número de opções entrada e saída, para o tipo de entrada que precisávamos, era nescessário utilizar os switches. Porém, o número de switches na placa era menor do que precisávamos para testar o circuito em sua totalidade. Optamos aqui por testar apenas os 4 bits menos signifcativos das entradas A e B, e mantivemos todos os demais.

Com a placa ajustada, efetuamos testes exaustivos para todas as operações implementadas. Os resultados foram todos dentro do esperado, levando em consideração nossa limitação de inputs à 4 bits.

\section{Conclusão}
Com estes dois projetos simples, tivemos um excelente primeiro contado com a linguagem VHDL, bem como à programação concorrente e desenvolvimento de circuitos FPGA. Os dois circuitos implementados (Multiplexador de Endereçamento e Unidade Lógica Aritimética) são blocos de construção chave para a maior parte dos circuitos complexos, e serão de suma importância não só para os demais trabalhos práticos que realizaremos ao longo do semestre, mas para nosso desenvolvimento acadêmico e profissional.

\end{document}
