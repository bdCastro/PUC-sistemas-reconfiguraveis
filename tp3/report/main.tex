\documentclass{article}

% Packages
\usepackage{geometry}
\usepackage{graphicx}
\usepackage{lipsum}
\usepackage{listings}
\usepackage[portuguese]{babel}

% -- Defining colors:
\usepackage[dvipsnames]{xcolor}
\definecolor{codegreen}{rgb}{0,0.6,0}
\definecolor{codegray}{rgb}{0.5,0.5,0.5}
\definecolor{codepurple}{rgb}{0.58,0,0.82}
\definecolor{backcolour}{rgb}{0.95,0.95,0.92}

% Definig a custom style:
\lstdefinestyle{mystyle}{
    backgroundcolor=\color{backcolour},   
    commentstyle=\color{codepurple},
    keywordstyle=\color{NavyBlue},
    numberstyle=\tiny\color{codegray},
    stringstyle=\color{codepurple},
    basicstyle=\ttfamily\footnotesize\bfseries,
    breakatwhitespace=false,
    breaklines=true,
    keepspaces=true,
    numbers=left,
    numbersep=5pt,
    showspaces=false,
    showstringspaces=false,
    showtabs=false,
    tabsize=2,
    captionpos=b
}

% -- Setting up the custom style:
\lstset{style=mystyle}

% Page setup
\geometry{a4paper, margin=2cm}

\begin{document}

% cover
\thispagestyle{empty} % Remove page number from first page

\begin{figure}[t]
    \includegraphics[width=3cm]{images/logo-puc-minas.png}
    \hspace{0.02\textwidth}
    \vline%
    \hspace{0.04\textwidth}
    \includegraphics[width=3cm]{images/logo-icei.jpeg}
\end{figure}

\hrulefill%
\vspace{\baselineskip}

\Large\noindent
\textbf{Pontifícia Universidade Católica de Minas Gerais} \\
\textbf{Instituto de Ciências Exatas e Informática} \\
\textbf{Departamento de Engenharia de Computação}

\begin{center}
    \vfill
    \Huge\textbf{Relatório do Laboratório 5} \\
    \vspace{0.5cm}
    \Large\textbf{Fonte Regulada com Diodo Zener} \\
    \vspace{1cm}
    \large \textbf{Professor}: Bruno Luiz Dias Alves de Castro \\
    \vspace{0.5cm}
    \large Bruno Luiz Dias Alves de Castro \\
    \large Bruno Luiz Dias Alves de Castro \\
    \large Bruno Luiz Dias Alves de Castro \\
    \vfill
    \large Belo Horizonte \\ Campus Coração Eucarístico \\
    \vspace{\baselineskip}
    \large \today
\end{center}

% table of contents
\newpage
\thispagestyle{empty}
\tableofcontents

% body
\newpage
\large % document text size

\section{Introdução}

Durante as aulas da disciplina de Sistemas Reconfiguráveis, fomos introduzidos à linguagem VHDL. VHDL (\textbf{V}HSIC \textbf{H}ardware \textbf{D}escription \textbf{L}anguage) é uma linguagem de descrição de hardware. Com ela, podemos montar circuitos lógicos de maneira totalmente textual, o que garante à linguagem uma grande vantagem ante à soluções visuais.

\subsection{Objetivos}

O objetivo deste terceiro trabalho prático é a implementação de uma memória ram e uma porta com barramento bidirecional. Essas estruturas serão utilizadas na construção interna do processador e para interfacear a comunicação interna e externa, respectivamente. Uma breve descrição de cada um é apresentada à seguir:

\subsubsection{port\_io}

O bloco port\_io é utilizada na comunicação entre a parte interno e externa do processador. É constituído de um barramento bidirecional configurado por um registrador nomeado tris\_reg, que define quais sinais serão entrada e saída.

\subsubsection{ram\_mem}

O bloco de ram\_mem, ou memória ram, são 4 registradores de propósito geral acessados através de endereços configurados na porta de endereçamento. Cada registrador é acessado por uma faixa de endereços, suportam acesso para escrita ou leitura.

\newpage

\section{port\_io}

O port\_io  é um circuito que possui uma estrutura de doiis registradores de 8 bits, e um barramento biderecional de entrada e saída, para intefaceamento com a estrutura. O registrados tris\_reg armazena o estado de cada bit (que poode ser entrada e saída), e o registrador port\_reg armazena dados, que podem ser escritos eplo usuário através do barramento de dados dbus\_in.

As entradas e saídas do circuito são descritas na tabela a baixo:\\

\begin{table}[ht]
    \begin{center}
        \begin{tabular}{|c|c|c|c|}
            \hline
            Nome & Tamanho & Tipo & Descrição\\
            \hline
            nrst & 1 bit & \textit{Input} & Entrada de \textit{reset} assíncrono.\\
            \hline
            clk\_in & 1 bit & \textit{Input} & Entrada de \textit{clock}.\\
            \hline
            abus\_in & 9 bits & \textit{Input} & Entrada de endereçamento para os registradores internos.\\
            \hline
            dbus\_in & 8 bits & \textit{Input} & Entrada de habilitação para escrita nos registradores\\
            \hline
            wr\_en & 1 bit & \textit{Input} & Entrada de habilitação de escrita.\\
            \hline
            rd\_en & 1 bit & \textit{Input} & Entrada de habilitação de leitura.\\
            \hline
            dbus\_out & 8 bits & \textit{Output} & Barramento de saída de dados, com 8 bits.\\
            \hline
            port\_io & 8 bits & \textit{Inout} & Porta bidirecional, com 8 bits.\\
            \hline
        \end{tabular}
    \end{center}
    \caption{Entradas e Saídas de port\_io}
\end{table}

Para o port\_io, também são utilizados 4 endereços internos, implementados através de estruturas \textit{GENERIC}, para enderaçamento dos registrados otris\_reg e port\_reg. São eles:

\begin{table}[ht]
\begin{center}
\begin{tabular}{|c|c|c|}
    \hline
    Nome & Endereço & Descrição \\
    \hline
    port\_addr & 0b00000011 & Especifica o endereço de escrita no registrador port\_reg. \\
    \hline
    tris\_addr & 0b00000111 & Especifica o endereço de escrita no registrador tris\_reg. \\
    \hline
    alt\_port\_addr & 0b10000000 & Endereço alternativo a port\_addr. \\
    \hline
    alt\_tris\_addr & 0b11000000 & Endereço alternativo a tris\_addr. \\
    \hline
\end{tabular}
\end{center}
\end{table}

\subsection{Implementação}

O circuitop port\_io foi implementado utilizando a linguagem VHDL.\\

O código na íntegra está abaixo:\\

\begin{lstlisting}[language=VHDL, caption={Código VHDL w\_reg}]
LIBRARY ieee;
USE ieee.std_logic_1164.all;
USE ieee.std_logic_unsigned.all;
USE ieee.numeric_std.all;

ENTITY port_io IS
    GENERIC (
        -- enderecos dos registradores port a tris
        port_addr: IN STD_LOGIC_VECTOR(8 DOWNTO 0) :=  "000000011";
        tris_addr: IN STD_LOGIC_VECTOR(8 DOWNTO 0) :=  "000000111";
        alt_port_addr: IN STD_LOGIC_VECTOR(8 DOWNTO 0) :=  "100000000";
        alt_tris_addr: IN STD_LOGIC_VECTOR(8 DOWNTO 0) :=  "110000000"
    );
    PORT (
        -- Processor Side
        -- Inputs
        nrst : IN STD_LOGIC;                            -- Reset
        clk_in: IN STD_LOGIC;                           -- Clock
        abus_in: IN STD_LOGIC_VECTOR(8 DOWNTO 0);       -- Enderecamento
        dbus_in: IN STD_LOGIC_VECTOR(7 DOWNTO 0);       -- Dados
        wr_en : IN STD_LOGIC;                           -- Enable escrita
        rd_en : IN STD_LOGIC;                           -- Enable leitura

        -- Outputs
        dbus_out : OUT STD_LOGIC_VECTOR(7 DOWNTO 0); -- Dados

        -- Port side
        port_io: INOUT STD_LOGIC_VECTOR(7 DOWNTO 0) -- bidirectional port
    );
END ENTITY;

ARCHITECTURE port_io OF port_io IS
    SIGNAL port_reg: STD_LOGIC_VECTOR(7 DOWNTO 0);
    SIGNAL tris_reg: STD_LOGIC_VECTOR(7 DOWNTO 0);
    SIGNAL latch: STD_LOGIC_VECTOR(7 DOWNTO 0);

    SIGNAL en_tris_addr: STD_LOGIC;
    SIGNAL en_port_addr: STD_LOGIC;
BEGIN
    -- verifica quais dos registradores se encontra no estado ativo
    en_tris_addr <= '1' WHEN (abus_in = tris_addr) OR (abus_in = alt_tris_addr) ELSE '0';
    en_port_addr <= '1' WHEN (abus_in = port_addr) OR (abus_in = alt_port_addr) ELSE '0';

    -- secao sequencial
    PROCESS(nrst, clk_in, abus_in, en_tris_addr, tris_reg, port_reg)
    BEGIN
        IF nrst = '0' THEN
            port_reg <= "00000000";
            tris_reg <= "11111111";
        -- parte sincroina
        ELSIF RISING_EDGE(clk_in) THEN
            IF en_tris_addr = '1' THEN
                IF(wr_en = '1') THEN
                    -- escrita
                    tris_reg <= dbus_in;
                END IF;
            END IF;
            IF en_port_addr = '1' THEN
                IF(wr_en = '1') THEN
                    -- escrita
                    port_reg <= dbus_in;
                END IF;
            END IF;
        END IF;
    END PROCESS;

    -- leitura
    dbus_out <= tris_reg WHEN en_tris_addr = '1' AND rd_en = '1' ELSE 
                latch WHEN en_port_addr = '1' AND rd_en = '1' ELSE "ZZZZZZZZ";

    - atualizacao do latch
    latch <= port_io WHEN en_port_addr = '0' OR rd_en = '0';

    -- altera os valores de port_io para 0 ou Z
    port_io(0) <= port_reg(0) WHEN tris_reg(0) = '0' ELSE 'Z';
    port_io(1) <= port_reg(1) WHEN tris_reg(1) = '0' ELSE 'Z';
    port_io(2) <= port_reg(2) WHEN tris_reg(2) = '0' ELSE 'Z';
    port_io(3) <= port_reg(3) WHEN tris_reg(3) = '0' ELSE 'Z';
    port_io(4) <= port_reg(4) WHEN tris_reg(4) = '0' ELSE 'Z';
    port_io(5) <= port_reg(5) WHEN tris_reg(5) = '0' ELSE 'Z';
    port_io(6) <= port_reg(6) WHEN tris_reg(6) = '0' ELSE 'Z';
    port_io(7) <= port_reg(7) WHEN tris_reg(7) = '0' ELSE 'Z';

END port_io;
\end{lstlisting}

\subsubsection{Descrição do funcionamento}

O circuito port\_io consiste de de duas partes: uma síncrona e uma assíncrona. A parte síncrona é composta pelas funções de \textit{reset} e escrita. Já a parte assícrona pelas funções de leitura e endereçamento.

Inicialmente, verifica-se se alguns dos registrados internos (port\_reg e tris\_reg) estão devidamente endereçados pela entrada abus\_in. Para isso, utilizamos um \textit{SIGNAL} auxiliar. O resultado será usado por ambas as partes síncrona e assíncrona.

Dentro do \textit{PROCESS}, a primeira função é a de \textit{reset} (ativo por nrst em baixo.), implementada de maneira assícrona (isto é, independente de uma borda de subida de \textit{clock}). Quando ocorrer, registrador port\_reg é zerado, e o tris\_reg tem todos os bits setados em `1'.

Ainda dentro do \textit{PROCESS}, são implementadas as funções de escrita sícrona. Após uma borda de subida de \textit{clock}, caso devidamente endereçado (feito anteriormente) e sinal wr\_en ativo, o registrados correspondente é escrito.

Fora do \textit{PROCESS}, de maneira assíncrona, são atualizadas a saída dbus\_out, com wr\_en ativo e enderaçamento correto, e o latch, caso não haja uma leitura de port\_reg.

Os bits de port\_io são setados individualmente a alta impedância caso estejam configurados como saída.

\subsection{Simulação}

Para testar nosso código VHDL e certificar-nos de que nosso circuito funciona de maneira esperada, simulamos alguns casos de testes utilizando o software Quatus II.\\

Os testes realizados foram os seguites:

\begin{enumerate}
    \item Testa exemplo do relatório.
    \begin{itemize}
        \item tris\_reg = ``00001111'' ou 0x0F;
        \item port\_io  = ``ZZZZ1111'' ou 0xZF;
        \item port\_reg = ``10101111'' ou 0xAF;
        \item \textbf{Comportamento esperado:}
        \begin{itemize}
            \item dbus\_out = ``10101111'' ou 0xAF;
            \item port\_io\_result =  ``10101111'' ou 0xAF;
        \end{itemize}
    \end{itemize}

    \item Teste com port\_io entradas e saídas alternadas.
    \begin{itemize}
        \item tris\_reg = ``10101010'' ou 0xAA;
        \item port\_io  = ``1Z1Z1Z1Z'';
        \item port\_reg = ``11111111'' ou 0xFF;
        \item \textbf{Comportamento esperado:}
        \begin{itemize}
            \item dbus\_out = ``11111111'' ou 0xFF;
            \item port\_io\_result =  ``10101010'' ou 0xAA;
        \end{itemize}
    \end{itemize}

    \item Teste com port\_io saídas e entradas alternadas.
    \begin{itemize}
        \item tris\_reg = ``01010101'' ou 0x55;
        \item port\_io  = ``Z1Z1Z1Z1'';
        \item port\_reg = ``11111111'' ou 0xFF;
        \item \textbf{Comportamento esperado:}
        \begin{itemize}
            \item dbus\_out = ``11111111'' ou 0xFF;
            \item port\_io\_result =  ``01010101'' ou 0x55;
        \end{itemize}
    \end{itemize}
\end{enumerate}

\begin{figure}[ht]
\begin{center}
    \includegraphics[width=15cm]{images/sim-port-io.png}
    \caption{Simulação do circuito port\_io}
\end{center}
\end{figure}

\newpage

\section{ram\_mem}

O bloco ram\_mem é um circuito que funciona como uma memória RAM. Ele é dividido em 4 blocos: 3 de 80 bytes, e 1 de 16 bytes. A escrita e leitura em cada um desses blocos é invisível para o usuário. Isto é, apenas um espaço de endereçamento é utilizado, e o circuito deve manejar em qual bloco escrever, ou ler, de acordo com a especificação.

\begin{table}[ht]
    \begin{center}
        \begin{tabular}{|c|c|c|c|}
            \hline
            Nome & Tamanho & Tipo & Descrição\\
            \hline
            nrst & 1 bit & \textit{Input} & Entrada de \textit{reset} assíncrono.\\
            \hline
            clk\_in & 1 bit & \textit{Input} & Entrada de \textit{clock}.\\
            \hline
            abus\_in & 9 bit & \textit{Input} & Entrada de enderençamento.\\
            \hline
            dbus\_in & 8 bits & \textit{Input} & Entrada de dados para escrita.\\
            \hline
            wr\_en & 1 bit & \textit{Input} & Entrada de habilitação de escrita.\\
            \hline
            rd\_en & 1 bit & \textit{Input} & Entrada de habilitação de leitura.\\
            \hline
            dbus\_out & 8 bits & \textit{Output} & Saída de dados hailitada por rd\_en.\\
            \hline
        \end{tabular}
    \end{center}
    \caption{Entradas e Saídas de ram\_mem}
\end{table}

A memória é divida em 4 blocos, com espaço de endereçamento compartilhado. Dependendo do endereço sendo lido/escrito, um bloco de memória diferente deve ser acessado, de acordo com a tabela abaixo:

\begin{table}[ht]
    \begin{center}
        \begin{tabular}{|c|c|}
            \hline
            Bloco & Faixa de endereçamento \\
            \hline
            mem0 & 020h $\sim$ 06Fh (80 bytes). Em decimal: 32 a 111 \\
            \hline
            mem1 & 0A0h $\sim$ 0EFh (80 bytes). Em decimal: 160 a 239 \\
            \hline
            mem2 & 20h $\sim$ 16Fh (80 bytes). Em decimal: 288 a 367 \\
            \hline
            mem\_com & 070h $\sim$ 07Fh (16 bytes). Em decimal: 112 a 127 \\
            \hline
        \end{tabular}
    \end{center}
    \caption{Entradas e Saídas de ram\_mem}
\end{table}

A área de memória mem\_com também pode ser endereçada através dos endereços 0F0h $\sim$ 0FFh,
170h $\sim$ 17Fh ou 1F0h $\sim$ 1FFh. Dessa forma os bits 8 e 7 de abus\_in não importam para o endereçamento
dessa área específica, sendo utilizados apenas os bits 6 a 0.

\subsection{Implementação}

O circuito ram\_mem foi implementado utilizando a linguagem VHDL.\\

O código na íntegra está abaixo:\\

\begin{lstlisting}[language=VHDL, caption={Código VHDL fsr\_reg}]
LIBRARY ieee;
USE ieee.std_logic_1164.all;
USE ieee.std_logic_unsigned.all;
USE ieee.numeric_std.all;

ENTITY ram_mem IS
    PORT (
        -- Inputs
        nrst : IN STD_LOGIC;                            -- Reset
        clk_in: IN STD_LOGIC;                           -- Clock
        abus_in: IN STD_LOGIC_VECTOR(8 DOWNTO 0);       -- Enderecamento
        dbus_in: IN STD_LOGIC_VECTOR(7 DOWNTO 0);       -- Dados
        wr_en : IN STD_LOGIC;                           -- Enable escrita
        rd_en : IN STD_LOGIC;                           -- Enable leitura

        -- Outputs
        dbus_out : OUT STD_LOGIC_VECTOR(7 DOWNTO 0)     -- Dados
    );
END ENTITY;

ARCHITECTURE ram_mem OF ram_mem IS
    TYPE mem_type0 IS ARRAY(0 TO 79) OF STD_LOGIC_VECTOR(7 DOWNTO 0);
    TYPE mem_type1 IS ARRAY(0 TO 15) OF STD_LOGIC_VECTOR(7 DOWNTO 0);

    SIGNAL mem0: mem_type0;
    SIGNAL mem1: mem_type0;
    SIGNAL mem2: mem_type0;
    SIGNAL mem_com: mem_type1;
    SIGNAL addr_int : INTEGER RANGE 0 TO 511;

BEGIN
    addr_int <= TO_INTEGER(UNSIGNED(abus_in));

    PROCESS(clk_in, nrst, addr_int, rd_en)
    BEGIN
        IF RISING_EDGE(clk_in) THEN
            IF wr_en = '1' THEN
                CASE addr_int IS 
                    -- mem0 80 bytes
                    WHEN 32 TO 111 =>
                        mem0(addr_int - 32) <= dbus_in;

                    -- mem1 80 bytes
                    WHEN 160 TO 239 =>
                        mem1(addr_int - 160) <= dbus_in;

                    -- mem2 80 bytes
                    WHEN 288 TO 367 =>
                        mem2(addr_int - 288) <= dbus_in;

                    -- mem_com 16 bytes
                    WHEN 112 TO 127 =>
                        mem_com(addr_int - 112) <= dbus_in;
                    WHEN 240 TO 255 =>
                        mem_com(addr_int - 240) <= dbus_in;
                    WHEN 368 TO 383 =>
                        mem_com(addr_int - 368) <= dbus_in;
                    WHEN 496 TO 511 =>
                        mem_com(addr_int - 496) <= dbus_in;

                    -- default
                    WHEN OTHERS =>
                END CASE;
            END IF;

        END IF;
        
        IF nrst = '0' THEN
            mem0 <= (OTHERS => (OTHERS => '0'));
            mem1 <= (OTHERS => (OTHERS => '0'));
            mem2 <= (OTHERS => (OTHERS => '0'));
            mem_com <= (OTHERS => (OTHERS => '0'));
        END IF;
    END PROCESS;

    PROCESS(clk_in, addr_int, rd_en, mem0, mem1, mem2, mem_com)
    BEGIN
        IF rd_en = '1' THEN
            CASE addr_int IS 
                -- mem0 80 bytes
                WHEN 32 TO 111 =>
                    dbus_out <= mem0(addr_int - 32);

                -- mem1 80 bytes
                WHEN 160 TO 239 =>
                    dbus_out <= mem1(addr_int - 160);

                -- mem2 80 bytes
                WHEN 288 TO 367 =>
                    dbus_out <= mem2(addr_int - 288);

                -- mem_com 16 bytes
                WHEN 112 TO 127 =>
                    dbus_out <= mem_com(addr_int - 112);
                WHEN 240 TO 255 =>
                    dbus_out <= mem_com(addr_int - 240);
                WHEN 368 TO 383 =>
                    dbus_out <= mem_com(addr_int - 368);
                WHEN 496 TO 511 =>
                    dbus_out <= mem_com(addr_int - 496);

                -- default
                WHEN OTHERS =>
            END CASE;
        ELSE
            dbus_out <= "ZZZZZZZZ";
        END IF;
    END PROCESS;
END ram_mem;     
\end{lstlisting}

\subsubsection{Descrição do funcionamento}

O circuito implementado possui duas partes: uma síncrona e outra assíncrona. A leitura é feita de forma síncrona, isto é, junto de uma borda de subida do \textit{clock}, já a leitura é feita de forma assícrona.

Primeiramente, se converte o sinal de entrada abus\_in para um inteiro, utilizando a função \textit{TO\_INTEGER()}. Com isso, podemos identificasr se o endereço sendo lido/escrito está dentro dos intervalos identificados.

Dentro da seção síncrona, após uma borda de subida de \textit{clock}, os dados são escritos no bloco correto, caso rd\_en esteja ativo. Já na parte assícrona, caso wr\_en esteja ativo, o valor endereçado é colocado na saída, acessado de acordo com o bloco utilizado.

O reset é feito de maneira assícrona, e possui preferência sobre a escrita, pis é feita de maneira processural depois dela.

\subsection{Simulação}

Para testar nosso código VHDL e certificar-nos de que nosso circuito funciona de maneira esperada, simulamos alguns casos de testes utilizando o software Quartus II.\\

Os testes realizados foram os seguites:

\begin{figure}[ht]
\begin{center}
        \includegraphics[width=15cm]{images/ram-mem-all.png}
        \caption{Simulação realizada com todos os blocos em conjunto.}
\end{center}
\end{figure}

\subsubsection{Registrador mem0}

\begin{enumerate}
    \item  Escrita e leitura com menor endereço possível.
    \begin{itemize}
        \item abus\_in = 32;
        \item dbus\_in = 24;
        \item \textbf{Comportamento esperado:}
        \begin{itemize}
            \item dbus\_out = 24;
        \end{itemize}
    \end{itemize}
    
    \item Testa leitura após reset.
    \begin{itemize}
        \item abus\_in = 32;
        \item dbus\_in = 0;
        \item \textbf{Comportamento esperado:}
        \begin{itemize}
            \item dbus\_out = 0;
        \end{itemize}
    \end{itemize}

    \item  Escrita e leitura com maior endereço possível.
    \begin{itemize}
        \item abus\_in = 111;
        \item dbus\_in = 66;
        \item \textbf{Comportamento esperado:}
        \begin{itemize}
            \item dbus\_out = 66;
        \end{itemize}
    \end{itemize}

    \item  Escrita e leitura com endereço aleatório.
    \begin{itemize}
        \item abus\_in = 72;
        \item dbus\_in = 99;
        \item \textbf{Comportamento esperado:}
        \begin{itemize}
            \item dbus\_out = 99;
        \end{itemize}
    \end{itemize}
\end{enumerate}

\begin{figure}[ht]
    \begin{center}
        \includegraphics[width=15cm]{images/ram-mem0.png}
        \caption{Simulação mem0}
\end{center}
\end{figure}

\subsubsection{Registrador mem1}

\begin{enumerate}
    \item  Escrita e leitura com menor endereço possível.
    \begin{itemize}
        \item abus\_in = 160;
        \item dbus\_in = 13;
        \item \textbf{Comportamento esperado:}
        \begin{itemize}
            \item dbus\_out = 13;
        \end{itemize}
    \end{itemize}
    
    \item Testa leitura após reset.
    \begin{itemize}
        \item abus\_in = 160;
        \item dbus\_in = 0;
        \item \textbf{Comportamento esperado:}
        \begin{itemize}
            \item dbus\_out = 0;
        \end{itemize}
    \end{itemize}

    \item  Escrita e leitura com maior endereço possível.
    \begin{itemize}
        \item abus\_in = 239;
        \item dbus\_in = 17;
        \item \textbf{Comportamento esperado:}
        \begin{itemize}
            \item dbus\_out = 17;
        \end{itemize}
    \end{itemize}

    \item  Escrita e leitura com endereço aleatório.
    \begin{itemize}
        \item abus\_in = 196;
        \item dbus\_in = 22;
        \item \textbf{Comportamento esperado:}
        \begin{itemize}
            \item dbus\_out = 22;
        \end{itemize}
    \end{itemize}
\end{enumerate}

\begin{figure}[ht]
    \begin{center}
            \includegraphics[width=15cm]{images/ram-mem1.png}
            \caption{Simulação mem1}
    \end{center}
    \end{figure}

\subsubsection{Registrador mem2}

\begin{enumerate}
    \item  Escrita e leitura com menor endereço possível.
    \begin{itemize}
        \item abus\_in = 288;
        \item dbus\_in = 1;
        \item \textbf{Comportamento esperado:}
        \begin{itemize}
            \item dbus\_out = 1;
        \end{itemize}
    \end{itemize}
    
    \item Testa leitura após reset.
    \begin{itemize}
        \item abus\_in = 288;
        \item dbus\_in = 0;
        \item \textbf{Comportamento esperado:}
        \begin{itemize}
            \item dbus\_out = 0;
        \end{itemize}
    \end{itemize}

    \item  Escrita e leitura com maior endereço possível.
    \begin{itemize}
        \item abus\_in = 367;
        \item dbus\_in = 17;
        \item \textbf{Comportamento esperado:}
        \begin{itemize}
            \item dbus\_out = 17;
        \end{itemize}
    \end{itemize}

    \item  Escrita e leitura com endereço aleatório.
    \begin{itemize}
        \item abus\_in = 333;
        \item dbus\_in = 49;
        \item \textbf{Comportamento esperado:}
        \begin{itemize}
            \item dbus\_out = 49;
        \end{itemize}
    \end{itemize}
\end{enumerate}

\begin{figure}[ht]
\begin{center}
        \includegraphics[width=15cm]{images/ram-mem2.png}
        \caption{Simulação mem2}
\end{center}
\end{figure}

\subsubsection{Registrador mem\_com}

\begin{enumerate}
    \item  Escrita e leitura com menor endereço possível.
    \begin{itemize}
        \item abus\_in = 112;
        \item dbus\_in = 24;
        \item \textbf{Comportamento esperado:}
        \begin{itemize}
            \item dbus\_out = 24;
        \end{itemize}
    \end{itemize}
    
    \item Testa leitura após reset.
    \begin{itemize}
        \item abus\_in = 112;
        \item dbus\_in = 0;
        \item \textbf{Comportamento esperado:}
        \begin{itemize}
            \item dbus\_out = 0;
        \end{itemize}
    \end{itemize}

    \item  Escrita e leitura com endereço 127.
    \begin{itemize}
        \item abus\_in = 127;
        \item dbus\_in = 66;
        \item \textbf{Comportamento esperado:}
        \begin{itemize}
            \item dbus\_out = 66;
        \end{itemize}
    \end{itemize}

    \item  Escrita e leitura com endereço 120.
    \begin{itemize}
        \item abus\_in = 120;
        \item dbus\_in = 100;
        \item \textbf{Comportamento esperado:}
        \begin{itemize}
            \item dbus\_out = 100;
        \end{itemize}
    \end{itemize}

    \item  Escrita e leitura com endereço 245.
    \begin{itemize}
        \item abus\_in = 245;
        \item dbus\_in = 37;
        \item \textbf{Comportamento esperado:}
        \begin{itemize}
            \item dbus\_out = 37;
        \end{itemize}
    \end{itemize}

    \item  Escrita e leitura com endereço 370.
    \begin{itemize}
        \item abus\_in = 370;
        \item dbus\_in = 123;
        \item \textbf{Comportamento esperado:}
        \begin{itemize}
            \item dbus\_out = 123;
        \end{itemize}
    \end{itemize}

    \item  Escrita e leitura com endereço 500.
    \begin{itemize}
        \item abus\_in = 500;
        \item dbus\_in = 8;
        \item \textbf{Comportamento esperado:}
        \begin{itemize}
            \item dbus\_out = 8;
        \end{itemize}
    \end{itemize}
\end{enumerate}

\begin{figure}[ht]
\begin{center}
    \includegraphics[width=15cm]{images/ram-mem-com.png}
    \caption{Simulação mem\_com}
\end{center}
\end{figure}

\section{Conclusão}

A implementação de registradores através de liguagens de descrição de hardware, como a VHDL, são maneiras poderosas de construir circuitos computacionais de maneira menos complexa e mais rápida.

Apesar de parecerem ``simples'' à primeira vista, sua implementação possuem nuancias que exigem a atenção do engenheiro, e testes exaustivos para certificar que funcionam como deveriam.

Neste trabalho prático, aprendemos como construir registradores e pilhas para auxiliar a contrução de nossos circuitos computacionais, utilizando conhecimentos da prática anterior e novas técnicas, como a utilização do ``PROCESS'', que permite a execução de trechos sequenciais no nosso circuito, e a tuilização do \textit{clock},  que permite a sincronização de rotinas no nosso cicuito.

Com os circuitos implementados, estamos mais confiantes nas nossas capacidades, e estamos um passo mais perto de implementar circuitos mais complexos como controladores e processadores.

\end{document}